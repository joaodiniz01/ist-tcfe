\section{Theoretical Analysis}
\label{sec:analysis}

In this section are shown the obtained results using a suitable theoretical model able to predict the output of the Envelope Detector and voltage Regulator circuits.

\subsection{Envelope detector}
The envelope detector restricts the voltage's amplitude and the voltage regulator decreases the ripple. As it was shown in the circuit's figure, the envelope detector is, basically, the resistance and the capacitor in parallel. Its main function is to decrease the ripple using the following expression:\par
\begin{equation}
    v_0(t) = Acos(\omega t_{off})^{\frac{-t+t_{off}}{RC}}
\end{equation}\par
Where: \par
\begin{itemize}
  \item A - Amplitude
  \item $\omega$ - angular frequency
  \item R - Resistance, constant obtained using the following expression:
\end{itemize}
\begin{equation}
    R = R_3 + 23r_d 
\end{equation}\par
($R_3$ and $r_d$ will be explained in the next section.)\par
\begin{itemize}
  \item C - Capacitance
  \item $t_{off}$ - Constant obtained using the following expression:
\end{itemize}
\begin{equation}
    t_{off}=\frac{1}{\omega}arctan(\frac{1}{\omega RC})
\end{equation}\par

\begin{figure}[H] \centering
\includegraphics[width=0.7\linewidth]{../mat/retified.pdf}
\caption{Rectified Signal.}
\label{fig:rectified}
\end{figure}

\begin{figure}[H] \centering
\includegraphics[width=0.7\linewidth]{../mat/envelope.pdf}
\caption{Envelope Detector Output Voltage.}
\label{fig:envelopeth}
\end{figure}

\subsection{Voltage Regulator}
In our circuit, the voltage regulator consists of 23 diodes in series that will impose the 12V voltage. The resistance in serie will decrease the amplitude.\par
The expressions used to do this were the following ones:\par
Sinusoidal part from envelope:\par
\begin{equation}
    v_{sr} = v_O - V_s
\end{equation}\par
Calculating $rd_{n}$ for all the diodes, where $rd$ is the resistance of each one:\par
\begin{equation}
    rd_n = 23rd
\end{equation}\par
And then, we have: \par
\begin{equation}
    vO_r = (\frac{rd_n}{rd_n + R_3})v_{sr}
\end{equation}\par
Where $R_3$ is the resistance in serie.\par
The expression used to obtain the vltage ripple was: \par
\begin{equation}
    v_{ripple} = maximum(V_{dc})-minimum(V_{dc})
\end{equation}\par
Where: \par
\begin{equation}
    V_{dc} = 21V_{on} + vO_r
\end{equation}\par
The results are shown below:\par

\begin{figure}[H] \centering
\includegraphics[width=0.7\linewidth]{../mat/outputdc.pdf}
\caption{DC Output Signal.}
\label{fig:outputdc}
\end{figure}

\begin{figure}[H] \centering
\includegraphics[width=0.7\linewidth]{../mat/v012.pdf}
\caption{Output Signal - 12 (deviation).}
\label{fig:v012}
\end{figure}

\begin{table}[H]
  \centering
  \begin{tabular}{|l|r|}
    \hline    
    {\bf Name} & {\bf Value [V]} \\ \hline
    \input{../mat/ripple_octave}
  \end{tabular}
  \label{tab:ripple}
\end{table}

\begin{figure}[H] \centering
\includegraphics[width=0.7\linewidth]{../mat/ripple.pdf}
\caption{Ripple.}
\label{fig:ripplegraph}
\end{figure}

