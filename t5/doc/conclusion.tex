\section{Conclusion}
\label{sec:conclusion}

Summing up, this laboratory provided us the opportunity to understand how a band pass filter using an OP-AMP circuit works with how each component relates to the band and the voltage gain.\par
Similarly to the previous lab assignment, there are some differences between the theoretical and simulation values obtained, even though they are really small, this is the case for the impedances, voltage gain and central frequency, for example. This can be explained by the non linear components used in this laboratory: the OP-AMP. In the first 2 lab assignments only linear components were used and because of that the simple theoretical analysis made matched perfectly the simulation analysis, as it should. This was no longer the case in this assignment. \par
As for the merit figure calculation, we used Spice's values as they provide the most accurate results. \par
Finally, we are going to compare the results from simulation and theoretical analysis side by side: \par
%compararações

\begin{center}
  \begin{tabular}{ | c | c | }
    \hline    
    {\bf Name} & {\bf Value} \\ \hline
    \input{m_tab}
  \end{tabular}
  \captionof{figure}{Merit Figure Table}
\end{center}

\begin{center}
  \begin{tabular}{ | c | c | }
    \hline    
    {\bf Name} & {\bf Value [$Hz$ or $dB$]} \\ \hline
    \input{conc_tab}
    \hline
  \end{tabular}
  \captionof{figure}{Central Frequency, Voltage Gain - Simulation}
\end{center}

\begin{center}
  \begin{tabular}{ | c | c | }
    \hline    
    {\bf Name} & {\bf Value [$Hz$ or $dB$]} \\ \hline
    \input{../mat/docImp2}
    \hline
  \end{tabular}
  \captionof{figure}{Central Frequency, Voltage Gain - Theoretical}
\end{center}

\begin{figure}[H]
    \minipage{0.45\textwidth}
      \includegraphics[width=\linewidth]{../sim/vo1f.pdf}
      \caption{Simulation Gain Frequency Response - $dB$}
    \endminipage\hfill
    \minipage{0.45\textwidth}
      \includegraphics[width=\linewidth]{../mat/fresponse1.pdf}
      \caption{Theoretical Gain Frequency Response - $dB$}
    \endminipage\hfill
\end{figure}

\begin{figure}[H]
    \minipage{0.45\textwidth}
      \includegraphics[width=\linewidth]{../sim/vo1p.pdf}
      \caption{Simulation Gain Frequency Response - Phase}
    \endminipage\hfill
    \minipage{0.45\textwidth}
      \includegraphics[width=\linewidth]{../mat/fresponse2.pdf}
      \caption{Theoretical Gain Frequency Response - Phase}
    \endminipage\hfill
\end{figure}

\begin{center}
  \begin{tabular}{ | c | c | }
    \hline    
    {\bf Name} & {\bf Value [$\Omega$]} \\ \hline
    \input{inimpedance_tab}
  \end{tabular}
  \captionof{figure}{Simulation Input Impedance}
\end{center}

\begin{center}
  \begin{tabular}{ | c | c | }
    \hline    
    {\bf Name} & {\bf Value [$\Omega$]} \\ \hline
    \input{outimpedance_tab}
  \end{tabular}
  \captionof{figure}{Simulation Output Impedance}
\end{center}

\begin{center}
  \begin{tabular}{ | c | c | }
    \hline    
    {\bf Name} & {\bf Value [$\Omega$]} \\ \hline
    \input{../mat/docImp3}
    \hline
  \end{tabular}
  \captionof{figure}{Theoretical Input and Output Impedances}
\end{center}



