\section{Simulation analysis}
\label{sec:simulation}
As said in the introduction, the first step to this laboratory assignment was to simulate a BPF using the script given by the professor in NGSpice.

\subsection{Frequency Response}
Right after simulating the circuit we output the frequency response of the same one both in $dB$ and phase. \par
The graphs we got are below: \par

\begin{figure}[H] \centering
\includegraphics[width=0.7\linewidth]{../sim/vo1f.pdf}
\caption{Frequency Analysis in $dB$}
\label{fig:frequency1}
\end{figure}

\begin{figure}[H] \centering
\includegraphics[width=0.7\linewidth]{../sim/vo1p.pdf}
\caption{Frequency Analysis - Phase}
\label{fig:frequency2}
\end{figure}

\subsection{Central Frequency in the Passband}
Firstly, we started our simulation analysis showing our results for the a requested value in the laboratory assignment which was the central frequency in the passband. This was achieved by using the given equation by the professor: \par
\begin{equation}
    CentralFrequency = \sqrt{Lower_cutoff Upper_cuttoff}
\end{equation}\par
In the images below we can see th output voltage plotted. The lower and upper cutoff and central frequency values obtained are also in the table below:

\begin{figure}[H] \centering
\includegraphics[width=0.7\linewidth]{../sim/vo1.pdf}
\caption{Output Voltage.}
\label{fig:output}
\end{figure}

\begin{table}[H]
  \centering
  \begin{tabular}{|l|r|}
    \hline    
    {\bf Name} & {\bf Value [$Hz$]} \\ \hline
    \input{central_tab}
  \end{tabular}
  \label{tab:central}
  \captionof{figure}{Central Frequency Gain}
\end{table}

\subsection{Output Voltage Gain}
After getting the central frequency in the previous subsection, in this section we will calculate the voltage gain in this frequency. \par
We did this by making the maximum of the difference of the out voltage in $dB$ and the in voltage in $dB$. \par
The value we got is in the table below:

\begin{table}[H]
  \centering
  \begin{tabular}{|l|r|}
    \hline    
    {\bf Name} & {\bf Value [$dB$]} \\ \hline
    \input{voltage_tab}
  \end{tabular}
  \label{tab:vgain}
  \captionof{figure}{Voltage Gain}
\end{table}

\subsection{Input and Output Impedances}
Lastly, we made NGSpice give us the input and output impedances. \par
The results we got are in the tables below: 

\begin{table}[H]
  \centering
  \begin{tabular}{|l|r|}
    \hline    
    {\bf Name} & {\bf Value [$\Omega$]} \\ \hline
    \input{inimpedance_tab}
  \end{tabular}
  \label{tab:inimpedance}
  \captionof{figure}{Input Impedance}
\end{table}

\begin{table}[H]
  \centering
  \begin{tabular}{|l|r|}
    \hline    
    {\bf Name} & {\bf Value [$\Omega$]} \\ \hline
    \input{outimpedance_tab}
  \end{tabular}
  \label{tab:outimpedance}
  \captionof{figure}{Output Impedance}
\end{table}




